\chapter{Fabrication \& Sensing of Input Devices}


\begin{quote}
The design of human-machine dialogues is, at least in part, the design of
artificial languages for [human-machine] communication. [W]e can analyze the
manipulations of an input device as sentences. Input devices are
those devices that allow some portion of these potential sentences actually to
be realized and communicated from human to machine.

--- Stuart Card, et al.

\end{quote}

This chapter outlines the multitude of digital fabrication techniques currently available, as well as potential characteristics of the artifacts that each process can create. We then align these properties to sensing techniques, describing a space of possible geometry/sensing links for input device design and fabrication.

\section{Definitions}

First, we briefly define uncommon words and machines that will be discussed in this chapter.

\emph{additive fabrication} : in additive fabrication, material is deposited and a shape is built up, leaving no waste

\emph{subtractive fabrication} : a subtractive fabrication process removes material to create a form. Excess material may be reused in another project or discarded.

\emph{3D printer} : a 3D printer is one of a class of machines that additively create a three-dimenstional model from one or more materials.

\emph{FFF} : FFF (fused-filament fabrication) 3D printers lay down material by melting and depositing a filament in a precise pattern.

\emph{model material} : model material is the substrate that the final object is made from

\emph{support material} : many modern 3D printers are capable of laying two types of materials, model material and a secondary, sacrificial material that can support overhangs in the model while printing, then be removed.

\emph{SLA} : SLA (stereolithography) printers use a bath of UV-curable polymer and a controllable UV laser. The laser "draws" each layer on the polymer, converting it from liquid to solid where it cures. Excess material is simply poured out for reuse.

\emph{SLS} : SLS (selective laser sintering) 3D printers contain a bed of material (e.g., metal powder) which is melted by a laser and rapidly cools into a solid, fused form. Excess material can be brushed off and reused.

\emph{PolyJet} : PolyJet printers have print heads similar to those of inkjet printers which sweep across the build area depositing material. Following the printer head is a UV light, which cures deposited material droplets.

\emph{vinyl cutter} : a vinyl cutter subtractively processes 2D materials with a 2-axis knife blade, cutting patterns into them. Vinyl cutters are typically used for thin, flexible materials.

\emph{laser cutter} : a laser cutter has a 2-axis laser for processing flat materials. Laser cutters can cut or etch into materials, and are often used for rigid, medium-thickness materials. Some have rotary attachments for etching on circular surfaces like the outside of a glass.

\emph{CNC router} : a CNC router uses a 2-axis rotary mill to cut through thick, rigid materials, like wood or certain metals. Some CNC routers are portable and can attach to many materials, while some are stationary with beds into which material is loaded.

\emph{CNC mill} : a CNC mill is a multi-axis machine which subtractively creates a 3D shape from a block of material, usually metal or wood.

\section{Digital Fabrication}

Digital fabrication machines are those which can take as input a digital design file, in 2D or 3D, and output a physical realization of that design; these machines stand in contrast to traditional crafting techniques (which take no design file) as well as manufacturing techniques (which require ``tooling'' for each design created). The true power of digital fabrication lies in its ability to create unique objects with each machine run \emph{without} the extensive setup and tooling necessary to change the product created by, for example, an injection moulding machine. This comes with the blessing and curse that each instance of an object costs as much to manufacture as the one before it, but allows for subtle variations between instances without additional cost. For example, doctors can now create custom 3D printed replacement joints that fit patients' bodies \cite{findref}.

The joint interests of industry, academia, and hobbyist makers have led to a flourishing ecosystem of digital fabrication and rapid prototyping (RP) machines. These machines describe a continuum from simple vinyl cutters that can create stickers to sophisticated multi-material 3D printers that can create multicolor and conductive designs that are fully 3D inside and out. These machines allow their manufactured products to achieve various structural and material properties. We examine these achievable properties, describing both those that are simple (and which can be created by $1$ machine and material) as well as those that are most sophisticated (requiring $3+$ materials or machines) (see Table \ref{table:properties}).

\valkyrie{maybe we just want to group by machine rather than sophistication? or we can group by property rather than machine? that is the story we actually want to tell.}

\begin{table}
\begin{center}
\begin{tabular}{rll}
\textbf{FEATURE}& \textbf{MANIFESTATION} & \textbf{POSSIBLE MACHINES} \\
\textbf{geometry} & $2D$ & vinyl cutter, paper cutter, inkjet printer \\
& $2.5D$ & laser cutter, CNC router \\
& $3D$ external & CNC mill, 3D printer \\
& $3D$ internal + external & 3D printer \\
\textbf{materials} & plastic & laser cutter, 3D printer \\
& metal & vinyl cutter, 3D printer, CNC mill \\
& wood & laser cutter, CNC router \\
& vinyl & vinyl cutter, laser cutter \\
& paper & paper cutter, laser cutter \\
\textbf{properties} & inert & \\
& adhesive & \\
& flexible & \\
& stiff & \\
& conductive & \\
& blended & \\
& multi-colored & \\
& transparent & \\
\end{tabular}
\end{center}
\caption{Digital fabrication enables fabrication of a wide variety of geometries with different material and structural properties. \valkyrie{I still hate this table. maybe just write the words first and match the table to it.}}
\label{table:properties}
\end{table}

\subsection{Geometry Fabrication}

Digital fabrication machines can support any complexity of geometry, from 2D images on paper (as an inkjet printer produces) to 3D projections of 4D objects (like Shapeways's Klein bottles printed in steel) (see Figure \ref{fig:range}. We describe the possibilities for the various geometries, as well as machines that could produce them. Note that we list machines at the edge of their range: for example, a CNC mill (listed under 3D external) can also make 2.5D or 2D objects.

\subsubsection{2D geometry}

2D geometry lies flat on a surface, but can manifest as an image printed on paper, a sticker cut from vinyl, or a barcode etched on granite. Machines that support 2D geometry fabrication include vinyl and paper cutters, as well as inkjet printers.

\subsubsection{2.5D geometry}

A slight jump from 2D is 2.5D: a 2D shape with additional \emph{depth} information. The canonical machines to create 2.5 objects are laser cutters (which can take multiple passes or vary laser 

\subsection{Simple Fabrication}
Beginning with simple processes, digital fabrication includes several types of machines, like inkjet printers, laser cutters, and vinyl cutters. We define ``simple fabrication'' as requiring only $1$ machine and $1$ material, but these materials can be as diverse as paper, wood, plastic, or metal. Although the physical requirements (machines and materials) for these types of artifacts are simple, their final properties can be quite complex. \valkyrie{shouldn't 3D printers be in here? single-material ones, anyway?}

\emph{Additive} fabrication methods, like laying down ink on paper or extruding plastic onto a bed, are 

\subsection{Moderate Fabrication}

\subsection{Sophisticated Fabrication}

\section{Single-Sensor Sensing Techniques}

The essence of Human Computer Interaction is a computer understanding a person's interactions. Thus, while many of the input techniques here could be used in, for example, machine-to-machine communication, we describe how a person's actions might create a usable control signal.

A ``single sensor'' can take many forms, ranging from a humble switch which opens and closes to a high-speed video camera which captures 2D visual information at 1000Hz to an accelerometer measuring G-forces in 3 directions. Card, et al., provide an analysis of the space of input devices based on what is sensed: position, $\Delta$position, angle, $\Delta$angle, force, $\Delta$force, torque, and $\Delta$torque \cite{card-input}. We use this to frame our discussion, but include additional senseable aspects: identity (of a user), bend/$\Delta$bend, and ???. We could also include, for example, \emph{intent}, which is the input of NLP voice systems like Siri, however these types of non-physical input are challenging to conceive of as a part of a physical input device and we therefore ignore them for now.

\begin{table}
\begin{center}
\begin{tabular}{rll}
\textbf{TECHNIQUE} & \textbf{SENSOR} & \textbf{COMPATIBLE PROPERTIES} \\
0D electrical & switches & ? \\
1D electrical & capacitance sensor, slider, hall effect, photocell, FSR, bend sensor, rangefinder & \\
2D electrical & arrays of 1D sensors & \\
3D electrical & accelerometer & \\
aural & mic & \\
vision & camera & \\
ML & & \\
\end{tabular}
\end{center}
\caption{There are techniques. We can use them. \valkyrie{need a better list of techniques. also need to organize better... :( could organize by "thing sensed": contact, distance, movement, acceleration, position, change in position... in fact, we can actually just schlep the whole table from design space of input devices??? holy shit, yes. do it. ask bjoern how permissions work. \cite{card-input}}}
\label{table:sensing}
\end{table}



\subsection{Machine Learning}

These sensing techniques require extensive use of machine processing on multiple input data sources: machine learning in this sense can involve active techniques such as creating classifiers swept capacitive \cite{sato-touche} or acoustic \cite{ono-touchandactivate,laput-acoustruments} frequencies; as well as passive approaches involving air \cite{slyper-softrobots} or 
