% !TEX root = thesis.tex
\begin{abstract}

% The text of the abstract goes here.  If you need to use a \section
% command you will need to use \section*, \subsection*, etc. so that
% you don't get any numbering.  You probably won't be using any of
% these commands in the abstract anyway.

Tangible input devices, like video game controllers and mice, demonstrably improve user speed and accuracy in input tasks. Compared to touchscreen-based interfaces, however, they are much more challenging to prototype and build. Rapid prototyping digital fabrication machines, such as vinyl cutters, laser cutters, and 3D printers, now permeate the design process for tangible input devices. Using these tools, designers can realize a new tangible design faster than ever. The catch is that these machines are not used to create the interaction in these interactive product prototypes: they merely create the shell, case, or body, leaving the designer to cobble together electronics for sensing a user's input. What are the most cost-effective, fast, and flexible ways of sensing rapid-prototyped input devices? We explore the link between digital models and their fabricated forms to probe this question. We describe the capabilities of modern rapid prototyping machines, linking these abilities to potential sensing mechanisms when possible. We plunge more deeply into three examples of sensing/fabrication links, building analysis and design tools that help users design, fabricate, assemble, and \emph{use} input devices sensed through these links. We demonstrate each technique's flexibility to be used for many types of input devices.

\end{abstract}
