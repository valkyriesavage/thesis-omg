\chapter{Fabrication \& Sensing}

\section{Digital Fabrication}

Digital fabrication machines are those which can take as input a digital design file, in 2D or 3D, and output a physical realization of that design; these machines stand in contrast to traditional crafting techniques (which take no design file) as well as manufacturing techniques (which require ``tooling'' for each design created). The true power of digital fabrication lies in its ability to create unique objects with each machine run \emph{without} the extensive setup and tooling necessary to change the product created by, for example, an injection moulding machine. This comes with the blessing and curse that each instance of an object costs as much to manufacture as the one before it, but allows for subtle variations between instances without additional cost. For example, doctors can now create custom 3D printed replacement joints that fit patients' bodies \cite{findref}.

The joint interests of industry, academia, and hobbyist makers have led to a flourishing ecosystem of digital fabrication and rapid prototyping (RP) machines. These machines describe a continuum from simple vinyl cutters that can create stickers to sophisticated multi-material 3D printers that can create multicolor and conductive designs that are fully 3D inside and out. These machines allow their manufactured products to achieve varied structural and material properties. We examine these achievable properties, beginning with those that are simple (and which can be created by $1$ machine and material) and moving towards those that are most sophisticated (requiring $3+$ materials or machines) (see Table \ref{table:properties}).

\begin{table}
\begin{center}
\begin{tabular}{|c|c|c|c|}
\hline
\textbf{ }& \textbf{simple} & \textbf{moderate} & \textbf{sophisticated} \\
\hline
\textbf{geometry} & $0D,1D,2D,2.5D$ & $3D$ external & $3D$ internal + external \\
\hline
\textbf{fabricated} & inert materials, & arbitrary 3D structure, & 2D texture on \\
\textbf{properties} & arbitrary 2D patterns, & flexible materials, &  3D surface,\\
 & conductive materials, & transparent materials & active components \\
 & adhesive materials, & & \\
 & flexible materials & & \\
\hline
\end{tabular}
\end{center}
\caption{Digital fabrication enables fabrication of a wide variety of geometries with different material and structural properties. We define ``simple processes'' as those which use only one machine and material. Moderately complex processes use 1-2 machines and 1-2 materials. Sophisticated processes require 3+ materials or machines. Some properties may be achievable through multiple means.}
\label{table:properties}
\end{table}

\subsection{Simple Fabrication}
Beginning with simple processes, digital fabrication includes several types of machines, like inkjet printers, laser cutters, and vinyl cutters. We define ``simple fabrication'' as requiring only $1$ machine and $1$ material, but these materials can be as diverse as paper, wood, plastic, or metal. Although the physical requirements (machines and materials) for these types of artifacts are simple, their final properties can be quite complex. \valkyrie{shouldn't 3D printers be in here? single-material ones, anyway?}

\emph{Additive} fabrication methods, like laying down ink on paper or extruding plastic onto a bed, are 

\subsection{Moderate Fabrication}

\subsection{Sophisticated Fabrication}

\section{Single-Sensor Sensing Techniques}

A ``single sensor'' can take many forms.

\begin{table}
\begin{center}
\begin{tabular}{|c|c|c|}
\hline
\textbf{simple} & \textbf{moderate} & \textbf{sophisticated} \\
\hline
0D electrical (switch closing), & 1D processed change (audio), & machine-learning \\
1D electrical (capacitance), & 2D processed change (video) & \\
2D electrical (arrays of 1D sensors), & & \\
3D electrical (accelerometer) & & \\
\hline
\end{tabular}
\end{center}
\caption{Sensing techniques can be bucketed into simple, moderate, and sophisticated. Simple techniques can use electrical thresholding. Moderately complex processes require some type of higher-level processing usually performed by a computer. Sophisticated techniques involve multiple input data sources and require extensive machine processing.}
\label{table:sensing}
\end{table}

\subsection{Simple Sensing}

\subsection{Moderate Sensing}

\subsection{Sophisticated Sensing}

Sophisticated sensing techniques require extensive use of machine processing on multiple input data sources: this category comprises techniques commonly called ``machine learning.'' Machine learning in this sense can involve active techniques such as swept capacitive \cite{sato-touche} or acoustic \cite{ono-touchandactivate,laput-acoustruments} frequencies; as well as passive approaches involving air \cite{squeezapulse?} or 
