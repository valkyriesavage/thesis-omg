\chapter{Future Work \& Concluding Remarks}

\begin{quote}
When I look into the future, it's so bright it burns my eyes.

--- Oprah Winfrey
\end{quote}

This thesis has presented techniques for leveraging and modifying digital geometry of objects to aid in creating fast, cheap, and flexible prototyping tools. We conclude with a restatement of our contributions, as well as pointers to future work.

%\section{Restatement of Contributions}

%We presented a design space describing how to interlink geometry and sensing for digitally-fabricated prototype objects, as well as a series of explorations in promising areas of the design space. Specifically, we contributed:

%\begin{enumerate}
\item Fabbing to sense: a model and sensing co-design technique which uses knowledge of a particular sensing paradigm to automatically modify digital design files before fabrication, allowing improved or training-free sensing of the fabricated prototype. We offer three exemplars of this technique: Midas, Lamello, and Sauron.
\item Midas, a method for automatically generating custom capacitive touch sensors---cut from adhesive-backed conductive foil---by synthesizing sensor pads and routing connections from a high-level graphical specification:
%which allows a designer to author a high-level graphical specification of an object---and from that creates custom-shaped, flexible capacitive touch sensors by synthesizing sensor pads, auto-routing connections, and generating instructions for assembly and use: 
\begin{enumerate}
    \item a design tool using this method to enable users to to fabricate, program, and share touch-sensitive prototypes
    \item an evaluation demonstrating Midas’s expressivity and utility to designers
    \end{enumerate}
\item Lamello, a technique using passive plastic tine structures, 3D printed at interaction points and with predictable vibrational frequencies, to create passive tangible inputs sensed via audio:
%a technique which integrates algorithmically-generated audio-producing tine structures into movable components, creating passive tangible inputs---sensed by a microphone which classifies the tine-generated audio---that do not require training examples for accurate sensing:
\begin{enumerate}
    \item a design pipeline which predicts tine frequencies (and an evaluation that they can be accurately predicted) and senses user manipulation of components in real time
    \item a discussion of information encoding techniques useful for this technique, and a series of scripts to generate parts utilizing these encodings
    \end{enumerate}
\item Sauron, a design tool enabling users to rapidly turn 3D models of input devices into interactive 3D printed prototypes where a single camera senses input: %\valkyrie{this is the clearest description. modify others to be like this!}
\begin{enumerate}
    \item a method for tracking human input on physical components using a single camera placed inside a hollow object
    \item two algorithms for analyzing and modifying a 3D model’s internal geometry to increase the range of manipulations that can be detected by a single camera.
    \item an informal evaluation of our implementation of these techniques usable on models constructed in a professional CAD tool.
    \end{enumerate}
\end{enumerate}

\section{Future Work}

Over the course the thesis, we have pointed to a variety of limitations and possible improvements per-project. In general, we see open sourcing the projects and making their prototypes more mobile to be interesting engineering and research tasks. All three projects use sensing types common to today's mobile smart phones (capacitive touch, microphone-based audio, and camera-based video), and additional engineering could better leverage those and other sensors already available to designers from within their environment.

Beyond per-project improvements, we have laid out a design space for linking geometry to sensing. Midas, Lamello, and Sauron each represent a single point in this design space, and fuller exploration of the space may lead to companion projects to those described here. In particular, the advent of multi-material polyjet 3D printers seems to open a wide variety of options for exploring colors, transparency, and flexibility as features in a sensing design space, and the Voxel8 \cite{voxel8} printer, which can lay arbitrary conductive materials and plastic together in a single pass, points to opportunities to explore induction or magnetism as sensing operations.

Overall, we recognize that there are several assumptions made by the projects presented in this dissertation. Namely, our projects leverage a \emph{single fabrication machine} for creating \emph{one prototype} at a time, which is \emph{hand-optimized} by a designer and sensed by a \emph{single} sensor. We discuss possibilities opened by willfully subverting each of these assumptions in turn.

\subsection{Cooperation among fabrication machines}

The projects described in this dissertation have largely focused on single fabrication machines working to create a finished prototype object. We also see opportunities for combining the abilities of several machines, whether to speed up the prototyping process or to investigate unique properties that allow exploration of the machines and sensors design space.

    \subsubsection{For celerity}

    While 3D printers allow for near-infinite flexibility in the forms that they are able to create, they still run very slowly. This can be compared to laser cutters, which offer significant speedups in exchange for only producing 2- or 2.5D prototypes (or limited 3D prototypes, see \cite{mueller-laserorigami}). Some prior work has investigated speeding up fabrication through integration of lasercut and 3D printed pieces \cite{beyer-platener}, or use of building blocks with 3D printed parts \cite{mueller-fabrickation}; however, these speedups do not make any use of their knowledge of the completed object post-fabrication.

    \subsubsection{For properties}
    
    Speed is a factor worth considering, but by leveraging multiple machines designers can additionally access a larger variety of properties. Plastic can offer a sturdy base with configurable haptics \cite{torres-hapticprint}, while inkjet-printed circuitry can provide a slide-in base for electronics. Lasercut or cnc-milled wood may pair well with delicate paper to create shape-changing interfaces \cite{yao-pneui}.

\subsection{Branching prototypes}

One important benefit of digital models of prototypes is that they become like code: they can be stored, shared, replicated, versioned, and unit-tested. Version-control website github \cite{github} in 2013 added a built-in viewer for STL files in user repositories on the site, offering a powerful tool for those who wished to version their physical designs. However, each design is hand-crafted by the designer. In the future, we would like to explore tools which can create likely \emph{spaces} of prototype designs given an initial seed from a designer, and which then allow testing multiple similar designs in parallel. Given the nature of the sensing performed by the toolkits presented, this type of small multiples testing should be as straightforward as re-attaching the sensing module to the new prototype.

\subsection{Machine-optimized prototype designs}

While describing a design space and smartly \emph{exploring} points in it to test in parallel could support a designer who has something in mind, another opportunity lies in automatic \emph{generation} of digital interfaces to suit particular people and/or tasks. This is similar in theory to Cogtool \cite{john-cogtool}, which allows designers of web applications to demonstrate tasks for their interfaces, then optimizes the interface to make completion of those tasks as quick as possible for end-users.

We believe that there is significant territory to be explored in modeling users' individual capabilities as relevant to tangible input devices, as well as understanding how to create optimal inputs devices suited to specific tasks.

    \subsubsection{To suit particular users}

    Individual users have wide-ranging abilities and preferences, especially when it comes to something as personal as the hands. In addition, mobility-impaired users may have special requirements for input devices. Traditional mass-produced input devices are designed to be comfortable for $95$\% of target users, however the advent of digital fabrication allows for one-off objects with no startup costs like those associated with tooling in traditional manufacturing. Thus, we can measure the capabilities of a single person (How large are Giorgia's hands? How far can Ethan bend his thumbs? How fast can Shiry pinch her fingers together?), and use our results to design for that person specifically. Such measurements might inform dimensions (e.g., overall size of an input device), locations (e.g., spacing between buttons), or even sensitivity (e.g., matched to user grip strength). This has seen a bit of exploration in the form of anatomical scanning for medical devices \cite{smakman-curatio}, but \emph{motion} is critical for interaction.

    \subsubsection{To suit particular tasks}
    
    Our lives contain a variety of general-purpose input devices, like the mice and keyboards that we find at our desks, or the video game controllers we use when we unwind. As described above, with digital fabrication machines we can imagine a future in which each input device is uniquely suited to a single task: we can ask questions about a task, and how a particular person approaches it (What abilities does Pat tend to use most when playing League of Legends? How often does Friedhelm scroll through documents while editing them, compared to insert/delete tasks?), and use these to inform the layout (e.g., more frequently-used functions close to dominant hand/fingers) or sensitivity (e.g., a lighter push will activate a time-critical function, versus a more substantial push necessary for an irreversible function) of input devices.
    
\subsection{Multi-sensor Units}

In the interest of reducing assembly time, all of our projects leverage a \emph{single} sensor in a single location. However, many commercially-available sensor bundles could offer additional types of data. For example, modern smartphones collect a wide array of sensors---e.g., microphone, camera, accelerometer, gyroscope, magnetometer---into a package that could easily be attached to a single point in a prototype; likewise, Texas Instruments' BLE SensorTag sports an ambient light sensor, humidity sensor, barometric pressure sensor, magnetometer, temperature sensor, and more \cite{ti-sensortag}. Affixing a block of multiple sensors to a single point adheres to the goal of assembly speed, and could allow for additional flexibility and precision in sensing.

\section{Closing Remarks}

As computing moves off the desktop and into the world, we see an exploration of many varieties of input devices designed to suit new and specific tasks. These new devices necessitate a new kind of prototyping, that does not rely fully on software but which can help create functional \emph{physical} objects in a fast, cheap, and flexible way. This dissertation has described digital fabrication as a means of accomplishing this: by linking a digital design to a completed physical object, we can shorten the drudgery associated with each iteration cycle. We have presented three examples of this: Midas performed sensor routing and used this knowledge to pre-program its capacitive sensor microcontroller and ensure assembly correctness. Lamello used 3D geometry to predict tines’ resonant frequencies for audio sensing. Sauron manipulated an object’s digital model to ensure that a single camera would be able to sense all the components inside, as well as to keep pieces from colliding in the hands of an end-user.

This suite of tools, and the space that they explore, will hopefully empower designers to invent and hone designs for the future of interaction.
