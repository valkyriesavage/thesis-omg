\begin{enumerate}
\item Fabbing to sense: a model and sensing co-design technique which uses knowledge of a particular sensing paradigm to automatically modify digital design files before fabrication, allowing improved or training-free sensing of the fabricated prototype. We offer three exemplars of this technique: Midas, Lamello, and Sauron.
\item Midas, a method for automatically generating custom capacitive touch sensors---cut from adhesive-backed conductive foil---by synthesizing sensor pads and routing connections from a high-level graphical specification. We also demonstrate a \emph{design tool} using this method to enable users to fabricate, program, and share touch-sensitive prototypes. Using our tool, we describe an evaluation demonstrating Midas's expressivity and utility to designers
\item Lamello, a technique using passive plastic tine structures, 3D printed at interaction points and with predictable vibrational frequencies, to create passive tangible inputs sensed via audio. We describe a \emph{design pipeline} which predicts tine frequencies (and an evaluation that they can be accurately predicted) and senses user manipulation of components in real time. We also include a discussion of information encoding techniques useful for this technique, and a series of scripts to generate parts utilizing these encodings.
\item Sauron, a design tool enabling users to rapidly turn 3D models of input devices into interactive 3D printed prototypes where a single camera senses input. We detail our method for tracking human input on physical components using a single camera placed inside a hollow object, and two algorithms for analyzing and modifying a 3D model's internal geometry to increase the range of manipulations that can be detected by a single camera. Finally, we describe an informal evaluation of our \emph{implementation} of these techniques usable on models constructed in a professional CAD tool.
\end{enumerate}