\chapter{Thesis Talk Video}

As part of UC Berkeley's graduation requirements, I gave a public talk about the contents of this thesis, representing an abridged version of the information contained in this document. This is available online: at time of writing, it is accessible as a YouTube video at \url{https://www.youtube.com/watch?v=SgZUxloEo4s}. It may move, but you'll almost certainly be able to find a link to it off of my homepage in the future at \url{http://valkyriesavage.com}.

\chapter{Definitions}

\begin{description}

\item[additive fabrication] In additive fabrication, material is deposited and a shape is built up.

\item[subtractive fabrication] A subtractive fabrication process removes material to create a form. Excess material may be reused in another project or discarded.

\item[3D printer] A 3D printer is one of a class of machines that additively create a three-dimenstional model from one or more materials.

\item[FFF] FFF (fused-filament fabrication) 3D printers lay down material by melting and depositing a filament in a precise pattern.

\item[model material] Model material is the substrate that composes the final object.

\item[support material] Many modern 3D printers are capable of laying two types of materials, model material and a secondary, sacrificial material that can support overhangs in the model during printing, then be removed.

\item[SLA] SLA (stereolithography) printers use a bath of UV-curable polymer and a controllable UV laser. The laser "draws" each layer on the polymer, causing photopolymerization where it strikes. Excess material is simply poured out for reuse.

\item[SLS] SLS (selective laser sintering) 3D printers contain a bed of material (e.g., metal powder) which is compacted and formed into a solid mass of material by heat and/or pressure without melting to the point of liquefaction. Excess material can be brushed off and reused.

\item[PolyJet] PolyJet printers have print heads similar to those of inkjet printers which sweep across the build area depositing material. Following the printer head is a UV light, which cures deposited material droplets.

\item[vinyl cutter] A vinyl cutter subtractively processes 2D materials with a 2-axis knife blade, cutting patterns into them. Vinyl cutters are typically used for thin, flexible materials.

\item[laser cutter] A laser cutter guides a laser's output over a 2D domain for processing flat materials. Laser cutters can cut or engrave into materials, and are often used for rigid materials $<\frac{1}{4}$ inch thick. Some have rotary attachments for engraving on circular surfaces like the outside of a glass.

\item[CNC router] A CNC router uses a 3-axis rotary mill to cut through thick, rigid materials, like wood or certain metals. Some CNC routers are portable and can attach to many materials, while some are stationary with beds into which material is loaded.

\item[CNC mill] A CNC mill is a multi-axis machine which subtractively creates a 3D shape from a block of material, usually metal or wood.

\end{description}

\chapter{Open-Sourced Code from Thesis}

I have open-sourced the code for the Midas and Sauron projects (available under the GPL 2.0), as well as this document and other projects---both research and personal---on my github account: \url{http://github.com/valkyriesavage}. Note that these pieces of software are ``research code'': at the time of publishing this document, I haven't fully cleaned and documented them for easy and straightforward use by others.